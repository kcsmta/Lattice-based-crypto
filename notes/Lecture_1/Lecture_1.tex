\documentclass[a4paper]{article}

\usepackage[english]{babel}
\usepackage[utf8]{inputenc}
\usepackage{amsmath}
\usepackage{graphicx}
\usepackage[colorinlistoftodos]{todonotes}

\title{Fully homomorphic encryption (FHE) schemes}

\begin{document}
	
	\maketitle
	\author{Khanh Nguyen}
	
	\section{Section 1}
	
	Voici le problème direct que nous allons être amenés à résoudre à chaque itération de notre algorithme:
	\begin{equation}
		\begin{cases}
			u_{t} - Lu = \tilde{c}.\chi^{-1}.u + g &\quad (x,t)\in Q \\
			u(x,0) = 0 &\quad x\in \Omega \\
			Bu = b(x,t) &\quad (x,t) \in S\\
		\end{cases}\\
	\end{equation}\\
	
	On utilise exactement les mêmes notations que dans Prilepko \& Kostin.
	\begin{figure}
		\centering
		\includegraphics[height=5cm]{./resources/figures/test.png}
		\caption{Thao Dien an cut}
	\end{figure}
	
	
	\section{Section 2}
	
	\subsection{Section 2.1}
	
	On rappelle la définition de l'opérateur A:
	\begin{equation}
		Ac = l(u_{t}(x,t,;c)) - L\chi - lg
	\end{equation}
	
	\subsection{Section 2.2}
	\begin{itemize}
		\item On initialise les constante $\Delta t$, $\delta>0$ et $T = n*\Delta t$, la variable $i=0$ et les fonctions $u^{0}=0$ , $c(x) < 0$ et $\chi(x)>0$.
		\item Tant que $\parallel lu^{n}-\chi \parallel_{\infty} > \delta$ : (on ignore la condition à la première itération)
		\begin{itemize}
			\item Pour $i$ de $0$ à $n-1$:
			\begin{itemize}
				\item On calcule $u^{i+1}$ par résolution du problème direct
			\end{itemize}
			\item $c \gets Ac$ pour $u=u^{n} $
		\end{itemize}
		\item On renvoie la dernière valeur de $c$.
	\end{itemize}
\end{document}